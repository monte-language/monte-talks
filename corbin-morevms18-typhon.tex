\documentclass[sigconf, 10pt]{acmart}

\usepackage{booktabs} % For formal tables


% Copyright
%\setcopyright{none}
%\setcopyright{acmcopyright}
%\setcopyright{acmlicensed}
\setcopyright{rightsretained}
%\setcopyright{usgov}
%\setcopyright{usgovmixed}
%\setcopyright{cagov}
%\setcopyright{cagovmixed}


%Conference
\acmConference{MoreVMs 2018}{April 2018}{Nice, France}
\copyrightyear{2018}


\begin{document}
\title{Basic Nanopass for RPython}


\author{Corbin Simpson}
\affiliation{%
  \institution{Matador Cloud LLC}
  \city{Portland}
  \state{Oregon}
  \country{USA}
}
\email{corbin@matador.cloud}


\begin{abstract}
The nanopass style of compiler design eases the task of maintaining
data-transformation passes in compilers, at the cost of requiring a
metaprogramming framework which aids in the generation of the boilerplate for
those passes. We present a Python 2 module for generating a nanopass compiler
pipeline for the RPython toolchain. Our approach uses standard Python
metaprogramming tactics to generate boilerplate RPython classes. The generated
passes use features of the RPython type system to enforce a modicum of
correctness, preventing certain common programmer errors. Passes also have
utility methods for raising error contexts to the end user and handling source
span information. Our module is about 200 lines of code and is used in the
Typhon compiler to optimize and interpret the Monte programming language.
\end{abstract}


\keywords{Monte, RPython, nanopass}


\bibliographystyle{ACM-Reference-Format}
\bibliography{corbin-morevms18-typhon}


\maketitle

\end{document}
